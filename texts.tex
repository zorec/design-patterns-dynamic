\documentclass[a4paper,10pt]{article}
%\documentclass[a4paper,10pt]{scrartcl}

\usepackage[utf8]{inputenc}

\title{Design patterns in dynamic languages}
\author{xbrand04@stud.fit.vutbr.cz}
\date{6.5.2015}

\pdfinfo{%
  /Title    (Design patterns in dynamic languages)
  /Author   (xbrand04@stud.fit.vutbr.cz)
  /Creator  ()
  /Producer ()
  /Subject  ()
  /Keywords (dynamic languages, design patterns, Ruby)
}

\begin{document}
\maketitle

Design patterns as they are defined and known from Gang of four book "Design Patterns: Elements of
reusable object-oriented software" represents ways to solve common problems in a software design.

In this text I am going to present various examples of design patterns in Ruby which
is great object-oriented representative of dynamic language. The emphasis is put on presenting
Ruby's dynamic features (like meta-programming) which are not possible in static languages. As a
result of these we will find out that design patterns can be easily expressed in dynamic languages.

% Each section starts with short description of design pattern followed by the specific example of
% problem and step by step solution.

\section{Iterator}
"The iterator pattern is used to provide a standard interface for traversing a collection of items
in an aggregate object without the need to understand its underlying structure."

Let us assume that we have an expression tree classes and we want to serialize them to access every
item of an expression.

Ruby provides two kinds of iterators to process elements in sequantial manner. There is internal
iterator which is defined and manipulated internally. External iterator can be passed around and
requires explicit control e.g. proceeding loop and moving items (calling next).

% \section{Interpreter}
% \section{Builder}


% TODO: quote from Peter N.
% Although examples might feel contrieved, ...

\end{document}

% TODO: bibliography
